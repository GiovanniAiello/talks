%!TEX root = ../main.tex
%-------------------------------------------------------------------------------
\section{Basic Model under Risk}
%-------------------------------------------------------------------------------

\begin{itemize}
\item[] \textbf{Ingredients}
\end{itemize}

These \textbf{ingredients} determine how we think about an agent's optimal decisions. In particular we will study how to think about \textit{optimal decision} in environment that are characterized by different assumptions about the information available to agent's information, i.e. study agent decisions under risk and ambiguity.\\\newline
%
Let's now establish some basic notation to describe these ingredients in a more formal way.
%-------------------------------------------------------------------------------
\begin{itemize}\item[] \textbf{Notation}\end{itemize}
%-------------------------------------------------------------------------------
More precisely, we will have a set of four alternatives: (1) Occupation A, (2) Occupation B, (3) School, and (4) Home. The time of agents in our model is set to 40 years, the discount factor is set to 0.95.

\begin{itemize}
\item When discussion rewards, preview observed and unobserved components.
\end{itemize}
%-------------------------------------------------------------------------------
\begin{itemize}\item[] \textbf{Decision Tree}\end{itemize}
%-------------------------------------------------------------------------------
In period 40, there are around 13,000 different nodes.
%-------------------------------------------------------------------------------
\begin{itemize}\item[] \textbf{Timing of Events}\end{itemize}
%-------------------------------------------------------------------------------

%-------------------------------------------------------------------------------
\begin{itemize}\item[] \textbf{Agent Characteristics}\end{itemize}
%-------------------------------------------------------------------------------
Agents are characterized by occupation-specific human capital, some learning by doing. As it turns out, at least some tasks will be similar in both occupations, so skills are at least partly transferable.
%-------------------------------------------------------------------------------
\begin{itemize}\item[] \textbf{State Space}\end{itemize}
%-------------------------------------------------------------------------------
Representation reflects fact of serially independence, initial conditions $x_{10}, x_{20} = 0, s_t = 10$. There is no depreciation of human capital.
%-------------------------------------------------------------------------------
\begin{itemize}\item[] \textbf{Agents' Objective under Risk}\end{itemize}
%-------------------------------------------------------------------------------
Agents act as to maximize the expected value of their discounted lifetime reward.
%-------------------------------------------------------------------------------
\subsection{Calibration}
%-------------------------------------------------------------------------------
Now, we will specialize the model even further. We will add functional form and distributional assumptions, and settle on a particular parametrization. The last figure shows the effect of schooling on wages for the two occupations. In \textit{Occupation A}, starting wages are higher but the returns to schooling are lower compared to \textit{Occupation B}. As agents accumulate more and more schooling at the beginning of their life-cycle, they are drawn towards \textit{Occupation B}. 
%-------------------------------------------------------------------------------
\begin{itemize}\item[] \textbf{Occupation A}\end{itemize}
%-------------------------------------------------------------------------------
The earnings equations are motivated by \citet{Mincer.1958,Mincer.1974}. Log earnings are linear in years of schooling, and linear and quadratic in years of labor market experience. The residual captures labor market luck. The rate of return to schooling is the same for all schooling levels. 
%-------------------------------------------------------------------------------
\begin{itemize}\item[] \textbf{Occupation B}\end{itemize}
%-------------------------------------------------------------------------------
\begin{itemize}\item[] \textbf{Wages and Experience}\end{itemize}
%-------------------------------------------------------------------------------
Schooling is set to ten years, which is the initial conditions all agents start out with. All agents start out with ten years. Let us briefly pause and compare the two types of occupations. ... Experience is set to their mean value in the sample, average education is 12 years. What do I know about the distribution of schooling. For how many is the 16 relevant? Occupation B is more skill intensive in the sense that schooling has a higher return and own experience has a higher return. Schooling an experience one provide general skill that is useful in both occupations, while occupation two only provides experience useful in Occupation B
%-------------------------------------------------------------------------------
\begin{itemize}\item[] \textbf{Wages and Schooling}\end{itemize}
%-------------------------------------------------------------------------------
Interaction between schooling and occupational choice. Schooling has a positive consumption value. occupation two has a lower mean wage at $t=0$. Experience is fixed at the mean values in the application.
%-------------------------------------------------------------------------------
\begin{itemize}\item[] \textbf{School}\end{itemize}
%-------------------------------------------------------------------------------
consumption Value and adjustment costs
%-------------------------------------------------------------------------------
\begin{itemize}\item[] \textbf{Shocks}\end{itemize}
%-------------------------------------------------------------------------------
idiosyncratic time-varying shocks
%-------------------------------------------------------------------------------
\begin{itemize}\item[] \textbf{Choices over Time}\end{itemize}
%-------------------------------------------------------------------------------
All agents start out identically, different choices over the life cycle are the cumulative effects of different shocks. Initially, 60\% increase their schooling but the share of agents in school in each period declines sharply. The share working in Occupation A starts to increase from 30\% and peaks out at about 60\% around period 15. Then declines back to 50\%. Occupation B increases continuously, initially only 2\% work in Occupation B but this share increases to about 40\%. Around 5\% stay at home each period.
