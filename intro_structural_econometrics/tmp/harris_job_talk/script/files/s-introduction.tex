%!TEX root = ../main.tex
%-------------------------------------------------------------------------------
\section{Introduction}
%-------------------------------------------------------------------------------
Welcome to \textit{Risk and Ambiguity in Educational Choices}. In my research agenda, I investigate the potential additional insights gained by introducing ambiguity in dynamic economic models. Currently, I work on \textbf{dynamic discrete choice models}.In these models, agents make repeated decisions over multiple periods. I study educational choices; starting at age eighteen agents each year decide whether to increase their education or work in the labor market.\\\newline
%
These \textbf{structural economic models} make explicit the agents' objective and the informational and institutional constraints under which they operate.\\\newline 
%
They allow to assess the relative importance of competing economic mechanisms that guide agents decisions and conduct ex ante evaluation of alternative policy proposals. Agents are \textbf{forward-looking} and thus take the future consequences of their immediate actions into account.  Their problem is \textbf{dynamic} as current investment into education, increases the rewards of future labor market participation. The agents operate in an \textbf{uncertain} economic environment, at least parts of their future payoffs are not known at the time of their decisions. For example think of it as labor market luck.\\\newline
%
The \textit{existing work on dynamic models of schooling} offers insights on the reasons for observed heterogeneity in educational attainment. The literature investigates (1) heterogeneity in returns to education, decomposing returns into benefits and costs of education, (2) selectively binding credit constraints, (3) heterogeneity in preferences (risk aversion, time preferences), and (4) the role of uncertainty.\\\newline
%
I hope to \textit{contribute to the literature} in the following way. The existing literature on these models studied \textbf{decisions under risk}. That is, agents know the exact distribution of the future random components. When face with a decision, they simply integrate out the random component and pursue the option with the higher expected value. I extend this literature by focusing on \textbf{decisions under ambiguity}.I instill a fear of model misspecification into the agents. Agents are not entirely sure about the distribution of the random components, but they try to make \textbf{robust decisions}. Decisions that work well under a variety of alternative forecasting models. They do not only consider one distribution, but a whole set. A make their decision by maximizing their worst-case outcome under the whole set of so called \textbf{admissible distributions}.\\\newline
%
\textbf{Why is this potentially productive extension?} In previous work with Prof. Heckman, we studied educational choices under risk in a standard version of the model. There we identified so called \textit{psychic cost} as a major factor in explaining educational enrollment patterns in the NLSY dataset. Another way to put it, these \textit{psychic cost} could be interpreted as effort costs. This datasets tracks educational choices of a cohort of individuals over time. Final educational attainment was simply too low compared to the returns in the data. However, pointing towards \textit{psychic costs} is very unsatisfactory as they remain as an essentially unexplained residual. As it turns out there is a \textbf{modeling trade-off}. If I fit a model where agents make decisions under risk, while in fact the economic environment is ambiguous, then this misspecification error shows up as \textit{psychic costs}. That is why I am now exploring ambiguity as a more interpretable economic mechanism leading to the observed patterns in the data.

Summing up, my goal for the next hour is the following: I hope to convince you that acknowledging ambiguity in dynamic models educational choice is (1) plausible, (2) meaningful, and (3) tractable.
%-------------------------------------------------------------------------------
\subsection{Starting point ...}
%-------------------------------------------------------------------------------
And this is how I intent to do it. The whole existing literature studying educational choices under risk was started and still relies on the key components outlined and developed in this paper. From a computational perspective solving these models boils down to a solving a finite horizon dynamic programming problem under risk. It involves several numerical challenges to make estimation of these models feasible. In particular function approximation and numerical integration.\\\newline
%
This basic model was of course extended to account for alternative mechanisms to account for the heterogeneity in educational attainment. However, all that remaining within the \textbf{paradigm of decisions under risk}.\\\newline
%
I am now branching off from this literature and study educational choices under ambiguity. All the basic ideas and challenges are part of this baseline model. However, a lot of the bells and whistles are absent and allow for a clear focus on this new mechanism. Of course, the final goal is clear. Have add all the mechanisms studied in the existing literature.\\\newline
%
The \textbf{final goal} is clear, of course. Fit a general model of educational choices under ambiguity to the NLSY that then allows to asses the relative importance of competing economic mechanisms proposed in the existing risk-based literature and conduct an ex ante evaluation of alternative policy proposals.\\\newline
%
Just as a word of caution, this is very much work in progress. In fact, it is the first time I am presenting this work. My only goal for the talk is to illustrate the challenges involved in introducing ambiguity in these models and show you my first steps in doing so. At the end, I hope we agree, that it is project worth continued efforts.