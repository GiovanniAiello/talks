%!TEX root = ../main.tex
%-------------------------------------------------------------------------------
\section{Ambiguity}
%-------------------------------------------------------------------------------


%-------------------------------------------------------------------------------
\subsection{Modeling Ambiguity}
%-------------------------------------------------------------------------------
\begin{itemize}\item[] \textbf{Set of Admissible Beliefs}\end{itemize}
%-------------------------------------------------------------------------------
Agents center their beliefs around a \textbf{baseline model} $\mathcal{N}_0$ and center their beliefs for the shock distribution and consider \textbf{local perturbations} around it. $\theta$ governs the size of the ambiguity set, if $\theta = 0$ then we are in the \textbf{special case} where the decision maker is acting under risk. The set is the \textbf{same for all agents} and remains constant over time, i.e. the uncertainty cannot be reduced over time. However, even though the set is the same for all agents, the relevant worst-case distribution still differs between agents.\\\newline
%
In effect, the decision makers gains nothing from by having future actions depend explicitly on past realizations of uncertainty. This leads to a separability that is crucial for establishing the robust counterpart of the Bellman recursion \citep{Iyengar.2005}. Applicability treats each state as very different, so the applicability might depend on the time horizon (e.g. seconds in financial data or years in occupational choice).
%
\begin{itemize}
\item This is going to be different than just having a different value for the intercept.
\item Concepts from decision theory: Consequentialism, Dynamic Consistency.
\end{itemize}
%-------------------------------------------------------------------------------
\begin{itemize}\item[] \textbf{Exploring Set of Admissible Beliefs}\end{itemize}
%-------------------------------------------------------------------------------
A key economic assumption is \text{rectangularity}. It is best interpreted in an adversarial setting, where the decision maker chooses its policy and an adversary observes it. The adversary then observes it and and selects the distribution that minimizes the rewards. It is a form of an independence assumption. The choice of distribution in a particular state does not restrict the choices of the adversary in the future. 
%
\begin{itemize}
\item This is going to be different than just having a different value for the intercept.
\item There is no belief heterogeneity.
\end{itemize}
%-------------------------------------------------------------------------------
\begin{itemize}\item[] \textbf{Exploring Expected Total Values}\end{itemize}
%-------------------------------------------------------------------------------
Why not just call them value functions? T - 1. Here ambiguity does not matter, late in life-cycle. But I choose this setup as it allows me to have a clear cut comparison between the cases. The disturbances are set to zero in the $T - 1$

We are looking at an agent at the end of his career in our model. It is in second to last period. He has worked most of his life. Accumulated worked for 9 years in Occupation A and 20 years in Occupation B, just one year of additional schooling.
%-------------------------------------------------------------------------------
\begin{itemize}\item[] \textbf{Preferences under Ambiguity}\end{itemize}
%-------------------------------------------------------------------------------
While there is agreement that maximization of expected reward is the right approach for decisions under risk, there is no such consensus for decision under ambiguity. Axiomatic decision theory can clarify the tradeoffs \citet{Stoye.2012}.

Minimax loss is the best known alternative to a the Bayes Approach. It evaluates decision rules by imputing a worst-case scenario as opposed to aggregating loss with respect to a prior.

Agents compute the expected utility with respect to each admissible probability measure and act as to maximize the expected value of their discounted lifetime reward under the worst-case scenario \citep{Maccheroni.2006,Gilboa.1989,Hansen.2007}.

\citet{Hansen.2001} distinguish between model and payoff uncertainty, ECB Error modeling

The next figure shows the share of individuals in school over time. Overall, investment in schooling declines as ambiguity increases. Embracing ambiguity can thus provide a more interpretable explanation for low enrollment rates of income-maximizing agents than the presence of large psychic costs investigated in \citet{Eisenhauer.2015b}.

The next figure shows the increase in average schooling (in percentage terms) for a \$500 tuition subsidy for different levels of ambiguity. The impact of the policy decreases as the level of ambiguity increases. In the case of ambiguous payoffs, agents do not account for the full value of the subsidy. They only adjust their worst-case evaluation.


The next figure documents the share of agents that end up in each of the two occupations in the last period for different levels of ambiguity. Agents reduce their schooling investments as ambiguity increases and thus less and less end up working in \textit{Occupation B}. 
%-------------------------------------------------------------------------------
\subsection{Understanding Economic Mechanism}
%-------------------------------------------------------------------------------
%-------------------------------------------------------------------------------
\begin{itemize}\item[] \textbf{Slide on Economic Mechanism}\end{itemize}
%-------------------------------------------------------------------------------



%-------------------------------------------------------------------------------
\subsection{Assessing Model Misspecification}
%-------------------------------------------------------------------------------
Now I am breaking away from the common setup, where the \textbf{econometrican and the agent share the same model}. Instead, I illustrate how our conclusions are potentially flawed when we the econometrican fit a model of educational choices under risk to data generated that recorded agent decisions and outcomes under ambiguity.
%-------------------------------------------------------------------------------
\begin{itemize}\item[] \textbf{Slide on Model Misspecification}\end{itemize}
%-------------------------------------------------------------------------------
