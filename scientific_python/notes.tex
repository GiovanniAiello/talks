\documentclass[a4paper,12pt,bold,leqno,fleqn,]{scrartcl}

\renewcommand{\baselinestretch}{1.3}\normalsize 
\newcommand{\vect}[1]{\mathbf{#1}}
\newcommand{\thin}{\thinspace}
\newcommand{\thick}{\thickspace}

\newcommand{\N}{\mathrm{N}} %Normal Distribution
\newcommand{\U}{\mathrm{U}} %Uniform Distribution
\newcommand{\D}{\mathrm{D}} %Dirichlet Distribution
\newcommand{\W}{\mathrm{W}} %Wishart Distribution
\newcommand{\E}{\mbox{E}}       %Expectation
\newcommand{\Iden}{\mathbb{I}}  %Identity Matrix
\newcommand{\Ind}{\mathrm{I}}   %Indicator Function

\newcommand{\var}{\mathrm{var}\thin}
\newcommand{\plim}{\mathrm{plim}\thin}
\newcommand{\cov}{\mathrm{cov}\thin}
\newcommand\indep{\protect\mathpalette{\protect\independenT}{\perp}}
\def\independenT#1#2{\mathrel{\rlap{$#1#2$}\mkern5mu{#1#2}}}


\usepackage[sans]{dsfont}
\usepackage[round,longnamesfirst]{natbib}
\usepackage{bm}                                                                                                 %matrix symbol
\usepackage{setspace}                                                                                       %Fußnoten (allgm. 
\usepackage{hyperref}
%Zeilenabstände)
\usepackage{threeparttable}
\usepackage{lscape}                                                                                         %Querformat
\usepackage[latin1]{inputenc}                                                                       %Umlaute
\usepackage{graphicx}
\usepackage{amsmath}
\usepackage{amssymb}
\usepackage{fancybox}                                                                                       %Boxen und Rahmen
\usepackage{appendix}
\usepackage{enumerate}
                          %EURO Symbol    
\usepackage{tabularx}   
\usepackage{longtable}                                                                                  %Mehrseitige Tabellen
\usepackage{color,colortbl}                                                                         %Farbige Tabellen
\usepackage[left=2cm, right=2cm, top=2cm, bottom=3cm]{geometry} %Seitenränder
%\usepackage[normal]{caption2}[2002/08/03]                                              %Titel ohne float - Umgebung    
\definecolor{lightgrey}{gray}{0.95} %Farben mischen
\definecolor{grey}{gray}{0.85}
\definecolor{darkgrey}{gray}{0.80}

\newcommand{\mc}{\multicolumn}
%\parindent0pt

\newtheorem{Definition}{Definition}
\newtheorem{Remark}{Remark}
\newtheorem{Lemma}{Lemma}
\newtheorem{Theorem}{Theorem}
\newtheorem{Excercise}{Excercise}
\newtheorem{Result}{Result}
\newtheorem{Proposition}{Proposition}
\newtheorem{Prediction}{Prediction}
\newtheorem{Solution}{Solution}
\newtheorem{Problem}{Problem}

\setlength{\skip\footins}{1.0cm}            
\deffootnote[1em]{1.1em}{0em}{\textsuperscript{\thefootnotemark}}                       
\renewcommand{\arraystretch}{1.05}

%-------------------------------------------------------------------------------
% Title
%-------------------------------------------------------------------------------
\title{\textit{Python} for Scientific Computing\thanks{For further information or questions and suggestions, please contact us at \href{mailto: info@policy-lab.org}{info@policy-lab.org}.}}
\author{ }
\date{ }
%-------------------------------------------------------------------------------
%-------------------------------------------------------------------------------
\begin{document}
\maketitle


\vspace{0.5cm}
\renewcommand{\baselinestretch}{1.3}\normalsize 

\pagenumbering{arabic}
\setcounter{page}{1}


\thispagestyle{empty}


%-------------------------------------------------------------------------------
\section{Why \textit{Python} for Scientific Computing}
%-------------------------------------------------------------------------------

\begin{itemize}
\item For beginners there really is no difference between version two and three of \textit{Python}.
\item \textit{Python} is the most popular coding language for teaching introductory computer science courses at top-ranked U.S. departments. Numerous online courses, lecture notes, and tutorials are readily available \href{http://www.fullstackpython.com/best-python-resources.html}{online}.
\item In the private sector, most recent results from \href{https://www.codeeval.com}{\textit{CodeEval}} point in the same direction.
\item \textit{Python} is used heavily used by tech companies (e.g. Google, Dropbox, etc.) and in the financial sector (e.g. AQR). When I and several other colleagues applied for jobs in finance, we were handed \textit{Python} coding challenges as part of their application process.
\item \textit{Python} is so simple to learn, a lot if books explicitly target kids.
\item Reliable interfaces to numerous language make for a good trade-off between developer and cycle time: (1) \href{https://www.r-project.org}{R}, (2) \href{www.mathworks.com}{MATLAB}, (3) \href{https://en.wikipedia.org/wiki/Fortran}{Fortran}, and (4) \href{ttps://en.wikipedia.org/wiki/C_(programming_language}{C},
\item \textit{Python as the Glue} and \textit{Batteries included}
\item Funding agencies such as the \href{http://www.sloan.org/}{\textit{Sloan Foundation}} (\$600,000 for \textit{Python for Economists}) and others are funding \href{http://jupyter.org/}{\textit{Project Jupyter}} with \$ 6 Mio.
\end{itemize}

%-------------------------------------------------------------------------------
\section{SciPy Stack}
%-------------------------------------------------------------------------------
\begin{itemize}

\item \textit{SciPy Library}, a collection of numerical algorithms and domain-specific toolboxes, including signal processing, optimization, statistics and much more
\item \textit{NumPy}, the fundamental package for numerical computation. It defines the numerical array and matrix types and basic operations on them
\item \textit{matplotlib}, a mature and popular plotting package, that provides publication-quality 2D plotting as well as rudimentary 3D plotting
\item \textit{pandas}, providing high-performance, easy to use data structures
\item \textit{SymPy}, for symbolic mathematics and computer algebra
\item \textit{IPython}, a rich interactive interface, letting you quickly process data and test ideas
\item \textit{nose}, a framework for testing Python code.
\end{itemize}

Depending on your particular specialization, these packages might be of additional interest to you.

\begin{itemize}
\item \textit{statsmodels}, a \textit{Python} module that allows users to explore data, estimate statistical models, and perform statistical tests.
\end{itemize}

\textit{statsmodels}, toghether with \textit{pandas}, is a potential replacement for the \textit{R}, just use \textit{rpy2} to call \textit{R} functions directly from \textit{Python}. All these packages are included in the \href{https://www.continuum.io/why-anaconda}{\textit{Anaconda Distribution}}.
%-------------------------------------------------------------------------------
\section{Basic Example}
%-------------------------------------------------------------------------------
\begin{itemize}
\item The \textit{Jupyter Notebook} is a web application that allows you to create and share documents that contain live code, equations, visualizations and explanatory text. Uses include: data cleaning and transformation, numerical simulation, statistical modeling, machine learning and much more.
\item The \textit{IPython} notebook works in your web browser, allowing you to document your computation in an easily reproducible form. See a notebook for
\citet{Reinhart.2010} as an example \href{http://nbviewer.jupyter.org/github/vincentarelbundock/Reinhart-Rogoff/blob/master/reinhart-rogoff.ipynb}{here}.
\end{itemize}
%-------------------------------------------------------------------------------
\subsection{Data Visualization }
%-------------------------------------------------------------------------------
\begin{itemize}
\item See the \href{http://matplotlib.org/gallery.html}{\textit{matplotlib Thumbnail Gallery}} for many and much more elaborate examples of data visualization.
\end{itemize}
%-------------------------------------------------------------------------------
\subsection{Statistical Analysis}
%-------------------------------------------------------------------------------
\begin{itemize}
\item We will fit an \textit{Ordinary Least Squares (OLS)} model using \href{http://statsmodels.sourceforge.net}{\textit{statsmodels}}. See the online documentation for a full list of the library's capabilities.
\end{itemize}
%-------------------------------------------------------------------------------
\section{Integrated Development Environment}
%-------------------------------------------------------------------------------
\begin{itemize}
\item For simple analysis the IPython Notebook or even the command line is sufficient. However, for more involved scientific programming. I found the use of an IDE very useful.
\item An integrated development environment (IDE) is a software application that provides comprehensive facilities to computer programmers for software development.
\item It provides numerous features that make coding easier such as name completion, renaming of methods and classes across multiple files, extracting methods, variables, parameters, static code analysis.
\end{itemize}

If we have time, we can explore a module-based implementation of our basic example together.
\begin{itemize}
\item Project Window, Structure Window, Code Completion, Code Wrap, Refactoring of nested method (e.g. extract \verb+simulate endogenous+, modify interface), Style Guide, Debugging, Distraction Free Mode, Presentation Mode
\end{itemize}

%-------------------------------------------------------------------------------
\section{Conclusion}
%-------------------------------------------------------------------------------
If you have any further questions or comments, please do not hesitate to let me know. I also welcome any suggestions for improving this lecture.

%-------------------------------------------------------------------------------
\section{Additional Resources}
%-------------------------------------------------------------------------------
Let us have a brief look at selected resources together.


%-------------------------------------------------------------------------------
\bibliography{../../_material/literature}
\bibliographystyle{apalike}
%-------------------------------------------------------------------------------
%-------------------------------------------------------------------------------
\end{document}
