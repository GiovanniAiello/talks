\input{../../_material/header}
%-------------------------------------------------------------------------------
% Title
%-------------------------------------------------------------------------------
\title{\textit{Python} for Scientific Computing\thanks{For further information or questions and suggestions, please contact us at \href{mailto: info@policy-lab.org}{info@policy-lab.org}.}}
\author{ }
\date{ } 
%-------------------------------------------------------------------------------
%-------------------------------------------------------------------------------
\begin{document}%!TEX root = ../main.tex
\title{Risk and Ambiguity in Educational Choices}
\author{Philipp Eisenhauer}

\date{\today}

\let\otp\titlepage
%\renewcommand{\titlepage}{\otp\addtocounter{framenumber}{-1}}


%-------------------------------------------------------------------------------
\section{Why \textit{Python} for Scientific Computing}
%-------------------------------------------------------------------------------

\begin{itemize}
\item For beginners there really is no difference between version two and three of \textit{Python}.
\item \textit{Python} is the most popular coding language for teaching introductory computer science courses at top-ranked U.S. departments. Numerous online courses, lecture notes, and tutorials are readily available \href{http://www.fullstackpython.com/best-python-resources.html}{online}.
\item In the private sector, most recent results from \href{https://www.codeeval.com}{\textit{CodeEval}} point in the same direction.
\item \textit{Python} is used heavily used by tech companies (e.g. Google, Dropbox, etc.) and in the financial sector (e.g. AQR).
\item \textit{Python} is so simple to learn, a lot if books explicitly target kids. 
\end{itemize}


%-------------------------------------------------------------------------------
\section{SciPy Stack}
%-------------------------------------------------------------------------------
\begin{itemize}

\item \textit{SciPy Library}, a collection of numerical algorithms and domain-specific toolboxes, including signal processing, optimization, statistics and much more
\item \textit{NumPy}, the fundamental package for numerical computation. It defines the numerical array and matrix types and basic operations on them
\item \textit{matplotlib}, a mature and popular plotting package, that provides publication-quality 2D plotting as well as rudimentary 3D plotting
\item \textit{pandas}, providing high-performance, easy to use data structures
\item \textit{SymPy}, for symbolic mathematics and computer algebra
\item \textit{IPython}, a rich interactive interface, letting you quickly process data and test ideas
\item \textit{nose}, a framework for testing Python code.
\end{itemize}

Depending on your particular specialization, these packages might be of additional interest to you.

\begin{itemize}
\item \textit{statsmodels}, a \textit{Python} module that allows users to explore data, estimate statistical models, and perform statistical tests. 
\end{itemize}

\textit{statsmodels}, toghether with \textit{pandas}, is a potential replacement for the \textit{R}, just use \textit{rpy2} to call \textit{R} functions directly from \textit{Python}. All these packages are included in the \href{https://www.continuum.io/why-anaconda}{\textit{Anaconda Distribution}}.
%-------------------------------------------------------------------------------
\section{Basic Example}
%-------------------------------------------------------------------------------
\begin{itemize}
\item The \textit{IPython} notebook works in your web browser, allowing you to document your computation in an easily reproducible form. See a notebook for 
\citet{Reinhart.2010} as an example \href{http://nbviewer.jupyter.org/github/vincentarelbundock/Reinhart-Rogoff/blob/master/reinhart-rogoff.ipynb}{here}.
\end{itemize}
%-------------------------------------------------------------------------------
\subsection{Data Visualization }
%-------------------------------------------------------------------------------
\begin{itemize}
\item See the \href{http://matplotlib.org/gallery.html}{\textit{matplotlib Thumbnail Gallery}} for many and much more elaborate examples of data visualization.
\end{itemize}
%-------------------------------------------------------------------------------
\subsection{Statistical Analysis}
%-------------------------------------------------------------------------------
\begin{itemize}
\item We will fit an \textit{Ordinary Least Squares (OLS)} model using \href{http://statsmodels.sourceforge.net}{\textit{statsmodels}}. See the online documentation for a full list of the library's capabilities.
\end{itemize}
%-------------------------------------------------------------------------------
\section{Integrated Development Environment}
%-------------------------------------------------------------------------------
\begin{itemize}
\item For simple analysis the IPython Notebook or even the command line is sufficient. However, for more involved scientific programming. I found the use of an IDE very useful.
\item An integrated development environment (IDE) is a software application that provides comprehensive facilities to computer programmers for software development.

If we have time, we can get going on the Getting Started Guide for Students (http://bit.ly/1WDDJny) together. 
\end{itemize}
%-------------------------------------------------------------------------------
%-------------------------------------------------------------------------------
\bibliography{/home/peisenha/office/workspace/bibliography/literature}
\bibliographystyle{apalike}
%-------------------------------------------------------------------------------
%-------------------------------------------------------------------------------
\end{document}

