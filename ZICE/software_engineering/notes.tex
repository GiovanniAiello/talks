	\input{../../_material/header}
%-------------------------------------------------------------------------------
% Title
%-------------------------------------------------------------------------------
\title{Software Engineering for Economists\thanks{For further information or questions and suggestions, please contact us at \href{mailto: info@policy-lab.org}{info@policy-lab.org}.}}
\author{ }\date{ } 
%-------------------------------------------------------------------------------
%-------------------------------------------------------------------------------
\begin{document}%!TEX root = ../main.tex
\title{Risk and Ambiguity in Educational Choices}
\author{Philipp Eisenhauer}

\date{\today}

\let\otp\titlepage
%\renewcommand{\titlepage}{\otp\addtocounter{framenumber}{-1}}


%-------------------------------------------------------------------------------
\section{Building Confidence in a Model}
%-------------------------------------------------------------------------------

\begin{itemize}
\item Computational models of socio-economic phenomena are a manifestation of our perceived knowledge about the underlying processes. They key question is how much confidence should we have in a particular model? 
\item  It turns out to be useful to structure such a discussion around three interrelated questions \citet{Council.2012}.
\item  Software Engineering encompasses the tools and methods for defining requirements for designing, programming, testing, and managing software. It is crucial to ensure that the computational implementation is a faithful representation of the original mathematical model \citet{Oberkampf.2010}. Thus, it is part of the verification step.
\item As an aside, for those interested in structural microeconometrics, we were lucky enough to have \href{http://www.economics.ox.ac.uk/Academic/michael-keane}{Prof. Keane} talk about the process of developing, estimating, and validating in the \href{http://bfi.uchicago.edu/events/computational-economics-colloquium}{\textit{Computation Economics Colloquium}}.
\item Basic software engineering allows frees cognitive resources that we an use to expand the set of possible economic questions we can address responsibly.
\item Computational implementation is part of the scholarship.
\end{itemize}
%-------------------------------------------------------------------------------
\section{Research Example}
%-------------------------------------------------------------------------------


%-------------------------------------------------------------------------------
\section{Running Example}
%-------------------------------------------------------------------------------
\begin{itemize}
\item For the rest of this lecture, we will use a small examples to illustrate ideas of different software engineering tools. However, we will also have a brief look how these tools are applied in the more complex setting of my current research. The online code repository is available \href{https://github.com/robustToolbox/package}{online}.
\end{itemize}

%-------------------------------------------------------------------------------
\section{Version Control}
%-------------------------------------------------------------------------------

\nocite{Bilschak.2016}

%-------------------------------------------------------------------------------
\section{Testing}
%-------------------------------------------------------------------------------
\begin{itemize}
\item To see these basic ideas in action, let us check out the testing harness for my current research project \href{https://github.com/robustToolbox/package/tree/master/development/tests}{online}.
\item Using bugs to define test cases ensures that they only need to be fixed once.
\end{itemize}


%-------------------------------------------------------------------------------
\section{Profiling}
%-------------------------------------------------------------------------------
\begin{itemize}
\item Now that we have a well designed and tested version of our code, it is time address any performance issues. We will profile our program by measuring the execution time of the program. 
\item Profiling tools also measure the time spend in each function allowing us to target our development efforts at particularly time-consuming parts of the code.
\item Studying the output directly can be rather tedious for large programs. That is when visualization tools turn out very useful. We build on \href{http://jiffyclub.github.io/snakeviz}{SNAKEVIZ}.
\item For even more advanced visualization, check out \href{https://github.com/pwaller/pyprof2calltree}{pyprof2calltree}. Tutorial for advanced visualization using \href{http://bit.ly/1SaXJgM}{KcacheGrind}.
\end{itemize}


%-------------------------------------------------------------------------------
\section{Continuous Integration Workflow}
%-------------------------------------------------------------------------------
\begin{itemize}
\item By running the testing harness early and often, bugs are caught closer to their creation. This makes debugging much easier. 
\item Scalability of research team is improved as basic quality assurance is automated.
\item The badges signal to your fellow researchers that we take your responsibilities as a developer of research software serious.
\item  Reliable work-flow increases own satisfaction.
\end{itemize}

%-------------------------------------------------------------------------------
\section{Best Practices}
%-------------------------------------------------------------------------------
\begin{itemize}
\item Iterative project development with only incremental addition of features. Testing harness ensures that old features are not broken.
\end{itemize}
%-------------------------------------------------------------------------------
%-------------------------------------------------------------------------------
\nocite{Judd.2011, Wilson.2014, Bourque.2014, Schlesinger.1979}
\bibliography{/home/peisenha/office/workspace/bibliography/literature}
\bibliographystyle{apalike}
%-------------------------------------------------------------------------------
%-------------------------------------------------------------------------------
\end{document}
